\chapter{Introduction}


Bab ini memberikan gambaran awal mengenai ruang lingkup, tujuan, serta pendekatan pembelajaran yang digunakan dalam buku ini. Sebagai pengantar, bab ini menjelaskan alasan mengapa MongoDB dan basis data NoSQL relevan dalam pengembangan aplikasi modern, sekaligus memperkenalkan profil pembaca yang menjadi sasaran, prasyarat pengetahuan yang dibutuhkan, serta lingkungan pengembangan yang disarankan. Selain itu, bab ini memaparkan struktur umum buku sehingga pembaca dapat memahami alur materi dari konsep dasar hingga topik tingkat lanjut. Dengan demikian, bab ini berfungsi sebagai fondasi konseptual dan navigasional yang membantu pembaca memulai perjalanan mempelajari MongoDB secara terarah dan sistematis.

\section{Latar Belakang}

Perkembangan aplikasi web dan layanan digital dalam beberapa tahun terakhir mendorong kebutuhan akan sistem penyimpanan data yang mampu menangani volume besar, kecepatan akses tinggi, serta struktur data yang bervariasi. Pola interaksi pengguna melalui perangkat mobile, Internet of Things (IoT), media sosial, dan layanan berbasis cloud menghasilkan data yang tidak selalu cocok dimodelkan dengan pendekatan relasional tradisional. Dalam konteks ini, basis data NoSQL, termasuk MongoDB, hadir sebagai salah satu pilar penting dalam arsitektur aplikasi modern dengan menawarkan fleksibilitas skema, kemampuan skalabilitas horizontal, dan integrasi yang baik dengan ekosistem pengembangan masa kini.

Buku ini disusun untuk membantu pembaca memahami MongoDB dari sudut pandang praktis dan konseptual. Mulai dari pengenalan konsep NoSQL dan karakteristik MongoDB, proses instalasi dan persiapan lingkungan, operasi CRUD dasar, perancangan data modeling, hingga teknik querying lanjutan dan strategi indexing untuk optimasi performa, pembaca diajak melihat bagaimana MongoDB digunakan dalam skenario aplikasi nyata, misalnya dalam konteks sistem sederhana seperti \texttt{toko\_online}.

\section{Tujuan dan Sasaran Pembaca}

Tujuan utama buku ini adalah memberikan landasan yang cukup agar pembaca mampu memahami peran MongoDB dalam ekosistem basis data modern dan menggunakannya secara efektif dalam pengembangan aplikasi. Setelah mempelajari isi buku, diharapkan pembaca dapat menjelaskan perbedaan umum antara pendekatan relasional dan NoSQL, melakukan operasi CRUD dasar di MongoDB, merancang skema dokumen yang sesuai dengan kebutuhan aplikasi, menulis query lanjutan termasuk Aggregation Pipeline, serta merancang indeks yang selaras dengan pola akses data untuk menjaga performa.

Buku ini ditujukan bagi mahasiswa dan praktisi yang sudah memiliki pemahaman dasar pemrograman dan sedikit pengalaman dengan basis data relasional. Contoh-contoh kode banyak menggunakan shell MongoDB (\texttt{mongosh}), antarmuka grafis seperti MongoDB Compass, serta bahasa pemrograman Python melalui pustaka \texttt{pymongo}. Namun, fokus utamanya tetap pada konsep yang dapat diterapkan lintas bahasa pemrograman.

\section{Prasyarat dan Lingkungan Pengembangan}

Agar dapat mengikuti pembahasan dengan nyaman, pembaca diharapkan memahami konsep dasar pemrograman seperti variabel, tipe data, dan struktur kontrol alur, serta mengenal istilah umum di basis data relasional seperti tabel, kolom, baris, dan kunci. Dari sisi teknis, buku ini mengasumsikan pembaca memiliki akses ke lingkungan pengembangan yang telah terpasang MongoDB Community Edition, dapat menjalankan service \texttt{mongod}, dan menggunakan \texttt{mongosh} untuk berinteraksi dengan server. Penggunaan MongoDB Compass sangat disarankan untuk mempermudah eksplorasi visual terhadap database dan koleksi. Integrasi dengan aplikasi ditunjukkan melalui Python dan \texttt{pymongo}, tetapi contoh tersebut dapat dengan mudah diterjemahkan ke bahasa lain.

Sepanjang buku, contoh dan latihan akan banyak menggunakan skenario \texttt{toko\_online} dengan koleksi seperti \texttt{produk}, \texttt{orders}, dan \texttt{users}. Pendekatan ini dipilih karena struktur e-commerce sudah cukup familiar dan kaya kasus: ada katalog produk, transaksi, pelanggan, serta data terkait seperti alamat pengiriman, kategori, dan tag. Dengan basis ini, pembaca dapat fokus memahami teknik penggunaan MongoDB tanpa harus terlebih dahulu mempelajari domain yang terlalu abstrak.

\section{Gambaran Umum Struktur Buku}

Buku ini dibagi menjadi beberapa bab yang saling berhubungan. Bab 2 memperkenalkan konsep NoSQL dan MongoDB. Di dalamnya, pembaca akan menemukan penjelasan singkat mengenai apa itu NoSQL, bagaimana karakteristik MongoDB sebagai basis data dokumen, serta kapan MongoDB layak dipilih atau justru kurang sesuai. Bab ini juga memandu proses instalasi MongoDB Community Edition, menjalankan dan mengelola service, memasang \texttt{mongosh} dan MongoDB Compass, serta melakukan koneksi dasar ke server MongoDB dari shell, antarmuka grafis, dan aplikasi Python.

Bab 3 berfokus pada operasi CRUD dan data modeling di MongoDB. Pembaca akan mempelajari cara menyisipkan, membaca, memperbarui, dan menghapus dokumen dengan berbagai variasi query, sekaligus melihat bagaimana struktur dokumen dapat dirancang agar sesuai dengan pola akses aplikasi. Perbedaan prinsip antara data modeling di SQL dan NoSQL diperkenalkan, diikuti dengan pembahasan mengenai embedded dan referenced data, representasi relasi satu-ke-satu, satu-ke-banyak, dan banyak-ke-banyak, serta beberapa praktik desain skema yang umum digunakan di MongoDB, termasuk penggunaan denormalisasi yang terkontrol.

Bab 4 melanjutkan pembahasan ke ranah advanced querying dan indexing untuk optimasi performa. Di bagian querying, pembaca diajak mendalami operator perbandingan dan logika, melakukan pencarian pada array dan embedded documents, serta memanfaatkan Aggregation Pipeline untuk analitik dan transformasi data langsung di sisi server. Di bagian indexing, dibahas bagaimana indeks dasar, compound index, dan text index dapat digunakan untuk mempercepat query; bagaimana menyelaraskan indeks dengan pola akses data; serta bagaimana menerapkan praktik terbaik indexing, termasuk memantau dan menganalisis query menggunakan \texttt{explain()}.

Bab penutup kemudian merangkum perjalanan materi, menyoroti kembali keterkaitan antara desain skema, pola query, dan strategi indexing dalam membangun aplikasi yang efisien dan skalabel. Daftar pustaka disajikan sebagai rujukan bagi pembaca yang ingin memperdalam pemahaman melalui dokumentasi resmi MongoDB, buku-buku lain, maupun sumber daring yang relevan. Dengan struktur ini, pembaca dapat membaca buku secara berurutan dari awal, atau melompat ke bab tertentu sebagai referensi saat mengerjakan proyek berbasis MongoDB.
